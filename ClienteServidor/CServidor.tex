\begin{document}
\section{What is a distributed system?}

Collection of independent computers that appear to the users of the system as a single system.

Examples: network of workstations, distributed manufacturing systems, and network of branch office computers

\begin{itemize}
\item Program
\item Process: what you get when you run the program
\item Message: used to communicate between processes.
\item Packet: Fragment of a message that might travel on a wirte
\item protocol: Formal description of message formats and the rules that two processes must follow in order to exchange those messages.
\item component: process or peace of hw required to run a process, support communication , etc
\item distributed system: app that executes a collection of protocols to coordinate the actions of multiple processes on a network that behaves as a system 
  
\end{itemize}

\section{Why do we build a distributed system?}

It is more fault tolerant. This is common in aeroplane systems, (the duplication or triplication of elements). But distributed systems are far more resilient to failure.

It is highly available, because it works even when some components are down.

Recoverable: Failing components can restart themselves and rejoin yhs system after solving the issue.

Consistent: The system can coordinate action by multiple components with concurrency and failure. That is what allows to a distributed system to act as a monolitc one.

Scalable: monolithic systems must grow all together. Instead, distributed systems can grow in each part without bothering others.

Predictable perfomance. Selfexplained.

Security: bad. When we distribute, it is less secure.

\section{All these are not true}
\begin{itemize}
\item The network is reliable
\item Latency is zero
\item Bandwidth is infinite
\item The network is secure
\item Topology doesn't change
\item There is one administrator
\item Transport cost is zero.
\item The network is homogeneous. 
\end{itemize}

\section{Client-Server applications}

The server provides a certain service, like a database queries, sending out stock prices, etc.

The client uses the service provided by the server, either displaying database query results to the user or making stock purchase recommendations to an investor.

It is common for these applications to be connected by a LAN.

\section{Cluster Computing}

Paralel programming, in which a single program is run in parallel on multiple machines. Each cluster is made by a collection of computer nodes that are controlled by a single master node. The master typically handles the allocation of nodes to a particular parallel program, maintains a batch queue of submitted jobs, and provides an interface for the users of the system (prime number looking counts with contributed time?: answ from prof: no, because the master does not handle all the nodes)

\section{Grid Computing}
No assumptions are made concerning similarity of HW, SO, networks, administrative domains, security policies, etc. There is a fabric layer that handles all the other components, the conectivity and the resource layer handle, well, all... they make a cluster, given the load. You have a load, the fabric layer gets the resources, get those available, loads them up, execute the task, destroy the cluster and makes the resources free again

\section{Cloud computing}

there is, like a grid computing at a bigger scale. It makes HW and SW disposable, as a service. First layer is infraestructure. I need 20 cpu and 20 tb of ram. The layer reserves that and makes you a `machine'. They can also give you all that, with a lot installed on it. I can also, hire software as a service,

In grid, we give it the task and the fabric layer discovers what it needs. In cloud, we say how much elements do we need.

\section{Architecture}

\subsection{Two Tier}
Client connects to the data sources directly. The logic remains on the client or the database.

It implies to divide the problem in two tiers. Concerns must be separated.

It is rather natural, we are used to apps working in these ways. It is easier to develop. It allows you to change, for example, to change a DB without affecting the user. If we add layers, it becomes slower. It is less secure.

With each abstraction, performance goes down and mantainability goes usually up. Portability is a bit bad because it depends on BD tech and stuff like that.

This is no longer used
\subsection{Three Tier}

Presentation Tier, Application Tier, Data Tier.

\section{Microservices}

Each element of functionality into a separate service. It allows for easier scaling

\section{ Versioning}

A commit is an irreversible change in a set of documents. We have a staging area where we add files. Then, the commits move the staging area to the repository. It makes them permanent. It also cleans the staging area into a working directory.

when we add a file, it goes to staged state, then if we commit, it goes to unmodified status. If we modify it, goes to modified, if we add, it goes back to staging area

You have origins. Those are places from where changes are received, or sent to.

The head is the latest revision on the repository. The master or trunk is the master location for code. There is where code is putted. The head does not change until we make a push.

Basic tasks: \textbf{Add} places a file under revision control. \textbf{Check in} it sends local chagnes to the repo. \textbf{Check out} Download files from the repo to your working copy. \textbf{Ignore} allows files to exists in your working copy but not in the repo. \textbf{ Revert} Throw away your working copy and restore last version.\textbf{Update/Sync} Updates your working copy to the last revision.

Diff: specific modification to a document
branch: generate duplicate of code
mrege

resolve: manual fixing of conflicted document changes
lock: bloquear cambios sobre archivos.

diff te muestra los cambios entre dos estados
\end{document}
