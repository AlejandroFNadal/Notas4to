\begin{document}

Numeros pseudoaleatorios:

Prueba de Medias
... This is boring to no end. Sorry, ale from the future. You will have to look for this elsewhere.

Chi-Square test:

check whether random numbers are uniformly distributed between 0-1.

Independence test: based on the next hypothesis:

$$h_0$$ : the numbers of set r are independent.
$$h_i$$ : the numbers of set r are not independent.

There are a few, like runs up and dows, runs up and down of the mean, poker tests, series test, holes tests.

One of the most used is the series test: it is related to chi-squared test.

Copy down this stuff later. Really. Is dead boring.

\section{Random Variables}

The behaviour of the system is reflected by its variables.

How can we determine the kind of distribution just with the dataset?

The most common method to detect the kind of distribution is Chi-Square

The method is:

At least, get 30 data points of the random variable to analyze.

Calculate the mean and variance of the data.

Create a histogram of m = sqrt(n) intervals, and the get the frequency in each interval

Establish a null hypothesis, through a probability distribution that adjust to the shape of the histogram. The observed data is explained by a suggested distribution. The alternate hypothesis denies that.

Infostat: helps to adjust automatically.

Then, we calculate the expected frequency, given the probability function that was proposed.

Then we calculated the trial statistic, with the sum of relative error as a chi value

Inverse transform method. It can be used to simulate random variables.

First, we define a density function that represents the variable to be modeled.

Then, we calculated the accumulated function F(x).

Then, we obtain an expresion for x, and obtain the accumulated inverted function F(x)~

Then, generate the random variable x, changing the values with pseudorandom numbers in the inverse accumulated function.

The denial method, is used to get the random variable numbers with the random numbers. We need two random values. 
\end{document}
