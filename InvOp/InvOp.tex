\documentclass[a4paper,12pt]{article}
\begin{document}
\title{Investigacion Operativa}
\author{Alejandro Nadal}
\maketitle

\section{What is operation research?}

There are certain problems in life that can be solved through operation research. Generaly, we try to improve certain quantifiable aspect through a decision.

These problems have a high combinatorial solution.

\section{Lean, applied to SW Ing}

First reference: Tom and Mary Poppendieck, they created the term Lean Software Development. The SW Agile manifesto is important as well:

\subsection{Values}
\begin{enumerate}
\item Individuals and relations above processes and tools
\item Working SW over excessive documentation
\item Colaboration with the client over contractual negotiations
\item Answers to change over following a plan
\end{enumerate}
\subsection{Principles}
\begin{enumerate}
\item Our biggest priority is to satisfy the client through early and continous delivery of valued software
\item We accept the change of requisites, even in late stages of development. Agile processes take advantage of change to give competitive advantage to the client
\item We deliver functional sw frecuently, in periods of two weeks to two months, aiming for shortest time span possible
\item Project owners and developers work together during all the project.
\item Projects are developed around motivated individuals. They must have the environment and support that they need, and they must be trusted with the work execution.
\item The most efficient method to communicate information to the development team and between its members is face to face talk
\item Functioning SW is the main measure of successe
\item Agile processes promove sustainable development. The promoters, developers and users must be capable of keeping a constant rythm for undefined periods of time
\item Continuous attention to technical excelence and good design, is what improves Agility.
\item Best architectures, requisites and designs come from self organized teams
\item The team reflects on how to be more effective in regular intervals, in order to adjust and perfect its behaviour.

\end{enumerate}

\subsection{Lean Software Development Principles}

\begin{enumerate}
\item Delete waste, everything that does not have value for the client
\item Amplify learning
\item Take decisions as late as possible, when we have more info
\item Deliver as fast as possible
\item Delegate responsability on the team.
\item Build with integrity.
\item Global vision: try to avoid local improvements, instead, go for a global approach.

\end{enumerate}

\subsection{Google ten principles}
\begin{enumerate}
\item Focus on the \textbf{user} and everything else will come
\item It is better to specialize in something and do it really well
\item Speed is a safe value
\item Democracy in web works: people help to choose what is better because of their own choices
\item You do not need to be in your office to get your answer. That is for users, you are supposed to be access to your technology everywhere
\item You can get income acting ethically
\item Information never ends, you can keep improving
\item The necesity of information overcomes every possible fronteer.
\item You can be a professional without a suit: Work can be fun and it can empower you. You must trust your own employees.
\item Do not conform just with excelent results. Predict future necesities from users.
  
\end{enumerate}

\section{People and Workgroups}


Poppendieck says the success of Lean depends on human behaviour. Four tools are necessary: Self-determination, motivation, leadership, and experience. People's work must have purpose. Managment becomes a facilitator, they must detect the needs of the team. The team must be united, be a group that trusts and competes with each other in a good way. Leaders must be trusted, respected. The leader must show how stuff is done, lead the way. The team is supposed to get experience on its own.

\paragraph{Issues}

Individualism, Breaking teams to frecuently (every project people are roteated, for example), and programmers belonging to many teams.

\subsection{How to incentive people}

In short, do not try to do that just with money. It does not last. There are two kinds of organizations, and only one works with lean.

\begin{itemize}
\item They just try to get money, individuals put effort and they are paid.
\item Try to stay in bussiness and give jobs and impact in industry. These employers give attention and effort and the organization tries to develop the individual to his/her maximum potential. The compromise goes both way. That is the way to incentive a person.
\end{itemize}

\section{ Continous Improvement}

Again, esentially, continous improvement of teams and people is necessary. Kaizen events are used to solve problems. These are a tool that gathers, in a team to the representatives of the different involved departments, for it to be solved. It does not last longer than a week and it is intensive work. 
\end{document}
